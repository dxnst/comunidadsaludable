\documentclass{article}
	\usepackage[utf8]{inputenc}
	\usepackage[T1]{fontenc}
	\usepackage{lmodern}

\title{Comunidad Saludable}

\begin{document}
\section{Primera Parte}
\subsection{Diagnóstico}
\paragraph{Infraestructura}
Aldea San José Chiquilajá se encuentra en el municipio de Quetzaltenango, en el departamento de Quetzaltenango, Guatemala. Se encuentra dentro de la cabecera municipal a 8 km del Centro Histórico de Quetzaltenango, y a 8.3 km del Centro de Salud de Quetzaltenango; a 8.7 km del Hospital Regional de Occidente y 7.7 km del Hospital General de Quetzaltenango del Instituto Guatemalteco de Seguridad Social; así mismo la aldea se encuentra a 192 km de la Ciudad de Guatemala.

Cuenta con un puesto de salud ubicado en el sector 1 de la aldea, justo arriba de la Iglesia del Milagroso Señor de Esquipulas y del mercado principal. El acceso principal a la aldea es mediante calle pavimentada, al igual que terracería que proporcionan vías de comunicación para el transporte de automóviles, camiones, y camionetas.

\paragraph{Número de viviendas, tipo de viviendas, servicios básicos}
Aldea San José Chiquilajá cuenta con un total de 1,502 viviendas que en su mayoría poseen paredes de adobe y techo de teja o lámina, una menor parte de las viviendas posee paredes de bloques de cemento y terraza.

Las viviendas cuentan en su mayoría con servicios básicos de agua, luz eléctrica, y drenaje. Recolección de basura, tratamiento de aguas?.

\paragraph{Número de personas por familia}
Aldea San José Chiquilajá cuenta con 10,511 habitantes según el último censo en 2,023. Se estima que las familias son conformadas por un promedio de 7 personas por vivienda.

\paragraph{Factores de riesgo en la comunidad}
\begin{itemize}
	\item Hacinamiento
	\item Inseguridad alimentaria
	\item Bajo nivel educativo
	\item Consumo excesivo de bebidas energizantes y alcohol
\end{itemize}

\subsection{Autoridades Locales}

\paragraph{Alcalde auxiliar}
La auxiliatura de la aldea de San José Chiquilajá está compuesta por los siguientes líderes comunitarios:
\begin{itemize}
	\item Alcalde: Tomas Chan Vicente
	\item Sindico: Oscar Domingo Ramos
	\item Regidor No. 1: Mario Chaj González
	\item Regidor No. 2: Lucas Fernando Mehia Hernández
	\item Regidor No. 3: Luciano Cos López
	\item Regidor No. 4: Salvador Alberto López
	\item Regidor No. 5: Juan Antonio Juárez López
	\item Regidor No. 6: Francisco Gaspar García González
	\item Regidor No. 7: Rosalio Máximo López Pérez
	\item Regidor No. 8: Gaspar Juárez González
	\item Auxiliar No. 1: Felipe Rogelio González García
	\item Auxiliar No. 2: Cesar Agusto Cortes Coy
	\item Auxiliar No. 3: Carlos Julio Hernández
	\item Auxiliar No. 4:
	\item Auxiliar No. 5:
	\item Auxiliar No. 6:
	\item Auxiliar No. 7:
	\item Auxiliar No. 8:
	\item Auxiliar No. 9:
	\item Auxiliar No. 10:
	\item Auxiliar No. 11:
	\item Auxiliar No. 12:
	\item Auxiliar No. 13:
	\item Auxiliar No. 14:
	\item Auxiliar No. 15:
	\item Auxiliar No. 16:
	\item Auxiliar No. 17:
	\item Auxiliar No. 18:
	\item Auxiliar No. 19:
	\item Auxiliar No. 20:
\end{itemize}

\paragraph{Consejos Comunitarios de Desarrollo Urbano y Rural}
\paragraph{Comités}
\paragraph{Comadronas}
\paragraph{Médicos mayas}
\paragraph{Otras personas que estén al servicio de la comunidad}

\subsection{Aspectos socioeconómicos}
\paragraph{Niveles de pobreza}
\paragraph{Nivel educativo}
\paragraph{Número de escuelas e institutos}
\paragraph{Áreas recreativas}
\paragraph{Producción}

\section{Segunda Parte}
\subsection{Evaluación Clínica de las Familias}
\paragraph{Evaluación Clínica}
\paragraph{Visitas Domiciliares}
\subsection{Identificación de factores de riesgo}
\paragraph{Mujeres embarazadas}
\paragraph{Mujeres puérperas}
\paragraph{Recién nacidos}
\paragraph{Desnutridos crónicos menores de 2 años}
\paragraph{Desnutridos agudos}
\paragraph{Persona BK+}
\paragraph{Niños no vacunados}
\paragraph{Otros: Personas que viven con VIH, diabetes, e hipertensión}
\paragraph{Elaboración de croquis de la comunidad}

\section{Tercera Parte}
\subsection{Descripción del Proyecto}
\paragraph{Listado Inicial de Problemas}
\paragraph{Priorización de Problemas}
\paragraph{Descripción del Proyecto}
\subsection{Otras Actividades}
\paragraph{}

\end{document}